\documentclass[../main.tex]{subfiles}

\begin{document}

\begin{itemize}
    \item Las tecnologías actuales nos permiten estudiar un amplio espectro de plasmas. Gracias a esto, se tuvo y se tiene un gran avance en distintas ramas de estudio no necesariamente relacionadas, como en la modificación de superficies materiales en la industria, o medicina. Además, dado que es un estado de la materia abundante en el universo, y presente en muchos lugares aquí en la tierra, el conocimiento generado nos permite entender aquellos fenómenos que nos rodean y tienen influencia sobre nosotros.
    \item Se definieron ciertos parámetros para poder describir las propiedades espaciales ($\lambda_D$) y temporales ($w_e$) de un plasma, así como también el parámetro $\Lambda$ para describir los fenómenos colectivos dentro del mismo. Estos surgen de la densidad y la temperatura asociada al plasma, y la cuasineutralidad característica de ellos. 
    \item Las reacciones de fusión nuclear se dan bajo condiciones específicas. Estas condiciones, para un plasma de fusión, requieren que los núcleos tengan la suficiente energía para acercarse y fusionarse.  Las densidades y temperaturas características juegan un papel importante. Bajo un sistema de contención del plasma, un proceso de reacción de fusión puede describirse en base a un balance entre la potencia de energía suministrada, y la potencia de energía perdida, mediante el criterio de Lawson.
    \item El estudio de la SOL (Scrape-Off-Layer), y de las regiones exteriores o de contorno del plasma en un reactor de fusión, es de gran importancia, pues estas regiones están más próximas a las superficies de contacto, las cuales cumplen algunas funciones fundamentales dentro del reactor. Producto de la interacción con las partículas cargadas, se produce una potencial flotante en estas superficies, y debido al efecto de apantallamiento en un plasma, se genera una zona de desbalance en la densidad de cargas, próxima a la superficie. La condición de Bohm nos dice que los iones ingresan a esta región a una velocidad mayor o igual a la velocidad del sonido en el plasma $c_s = k_B(T_e+T_i)/m_i$, acelerando a través de una región de transición entre el plasma neutro y la región de desbalance, o sheath.
    \end{itemize}

\end{document}
